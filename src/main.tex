%! Author = fabian.sigfridsson
%! Date = 2023-09-19

\documentclass[11p]{article}
\usepackage{amsmath}
\usepackage{graphicx}
\usepackage{fancyheadings}
%\usepackage{hyperref} % Fixar klickbara länkar
\usepackage[swedish]{babel}
\usepackage[
    backend=biber,
    style=authoryear-ibid,
    sorting=ynt
]{biblatex}
\usepackage[utf8]{inputenc}
\usepackage[T1]{fontenc}
\usepackage{hyperref}
% Källor
% \addbibresource{mall.bib}
% Sökväg till bildmappen
\graphicspath{ {./images/} }

\def\name{Fabian Sigfridsson}
\def\email{fabian.sigfridsson@elev.ga.ntig.se}
\def\foottitle{Mall Enkel raport}
\title{Labbrapport - Kaströrelse \\ \small Fysik}


\author{\name}
\date{\today}

\begin{document}

    \maketitle

% \section{Abstrakt} om det är mycket text
% \tableofcontents om det är mycket text

    \section{Syfte och frågeställning}
    Målet med laborationen är att kunna ange nedslagsplatsen för en kastbana med rimlig
    feluppskattning.
    Syftet är förstå begreppet kaströrelse och de modeller som används för att lösa
    problem som innehåller kaströrelse.
    \section{Del 1}
        \subsubsection{Bakgrund}
            Vi ska ta reda på mynningshastigheten från kanonen.

        \subsubsection{Metod}
            Vi kan ta fram hastigheten genom först ta reda på hur högt
            den skjuter rakt upp.
            Sedan använder vi formeln för en kaströrelses stigningshöjd
            och räknar ut hastigheten som kräv för den höjden.

        \subsubsection{Resultat}
            Bollen kom 127cm högt upp i luften.

        \subsubsection{Beräkningar}
            \begin{equation}
                y_{max}=\frac{v_0 \cdot \sin^2 \alpha}{2g}
                \label{eq:eq1}
            \end{equation}
            \begin{equation}
                v_0=\frac{y_{max} \cdot 2g}{\sin^2 \alpha}
                \label{eq:eq2}
            \end{equation}
            \begin{equation}
                y_{max}=1,27 m
                \label{eq:eq3}
            \end{equation}
            \begin{equation}
                \alpha=\frac{\Pi}{2}
                \label{eq:eq4}
            \end{equation}
            \begin{equation}
                g=9,82
                \label{eq:eq5}
            \end{equation}
            \begin{equation}
                v_0=\frac{1,27 \cdot 2 \cdot 9,82}{1}=5,03 m/s
                \label{eq:eq6}
            \end{equation}
        \subsubsection{Analys}
            Vi vet nu utgångshastigheten av kulan från kanonen är 5,03 m/s

    \section{Del 2}
        \subsubsection{Bakgrund}
            Nu ska vi
    \section{Del 3}

    % \newpage
    % \printbibliography
\end{document}